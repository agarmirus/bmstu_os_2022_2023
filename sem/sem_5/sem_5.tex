\documentclass[a4paper, 12pt]{report}

\usepackage[T2A]{fontenc}
\usepackage[utf8x]{inputenc}
\usepackage[english,russian]{babel}
\usepackage{amssymb,amsfonts,amsmath,mathtext,cite,enumerate,float}
\usepackage{pgfplots}
\usepackage{graphicx}
\usepackage{tocloft}
\usepackage{listings}
\usepackage[labelsep=period]{caption}

\makeatletter
\renewcommand{\@biblabel}[1]{#1.}
\makeatother

\usepackage{titlesec}

\usepackage{geometry}
\geometry{left=3cm}
\geometry{right=1cm}
\geometry{top=2cm}
\geometry{bottom=2cm}

\renewcommand{\theenumi}{\arabic{enumi}}
\renewcommand{\labelenumi}{\arabic{enumi}}
\renewcommand{\theenumii}{.\arabic{enumii}}
\renewcommand{\labelenumii}{\arabic{enumi}.\arabic{enumii}.}
\renewcommand{\theenumiii}{.\arabic{enumiii}}
\renewcommand{\labelenumiii}{\arabic{enumi}.\arabic{enumii}.\arabic{enumiii}.}

\renewcommand{\cftchapleader}{\cftdotfill{\cftdotsep}} % for chapters

\usepackage{indentfirst}

\titleformat{\chapter}[hang]{\Huge\bfseries}{\thechapter.}{1ex}{\Huge\bfseries}

\graphicspath{{images/}}%путь к рисункам

\newcommand{\floor}[1]{\lfloor #1 \rfloor}

\lstset{ %
	language=caml,                 % выбор языка для подсветки (здесь это С)
	basicstyle=\small\sffamily, % размер и начертание шрифта для подсветки кода
	numbers=left,               % где поставить нумерацию строк (слева\справа)
	numberstyle=\tiny,           % размер шрифта для номеров строк
	stepnumber=1,                   % размер шага между двумя номерами строк
	numbersep=5pt,                % как далеко отстоят номера строк от подсвечиваемого кода
	showspaces=false,            % показывать или нет пробелы специальными отступами
	showstringspaces=false,      % показывать или нет пробелы в строках
	showtabs=false,             % показывать или нет табуляцию в строках
	frame=single,              % рисовать рамку вокруг кода
	tabsize=2,                 % размер табуляции по умолчанию равен 2 пробелам
	captionpos=t,              % позиция заголовка вверху [t] или внизу [b] 
	breaklines=true,           % автоматически переносить строки (да\нет)
	breakatwhitespace=false, % переносить строки только если есть пробел
	escapeinside={\#*}{*)}   % если нужно добавить комментарии в коде
}

\pgfplotsset{width=0.85\linewidth, height=0.5\columnwidth}

\frenchspacing

\begin{document}
	\section*{Inter Process Communication}
	
	Возникновение двух коммерчески значимых линий: System5 и UNIX BSD. Они были приватизированы фирмами. В результате возникла ситуация: ПО для System5 нельзя было выполнять в UNIX BSD и наоборот. То есть, поскольку данные системы стали лицензионными, разрабам пришлось покупать лицензии. Возникла некоторая монополия. Пишут, что сообщество программистов обратилось с идеей создания станадарта, но все равно: POSIX - государственный стандарт.
	
	POSIX (Portable Operating System Interface for UNIX) перечисля более 1000 системных вызовов, обеспечивая тем самым переносимость ПО.
	
	!!! POSIX оговаривает только системные вызовы. Структура ядра и его функции не оговорены.
	
	\textbf{Факт:} IBM, когда разрабатывалась серия машин IBM360, разделило стоимость <<харда>> и <<софта>>. В это же время в СССР была другая тенденция: поднять цену. А у <<капиталюг>> - наоборот. Почему? На самом деле на рынке важно количество продаж.
	
	FIPS (Federal Information Processing Standard) - [кем разработан]
	
	POSIX.1 был разработан в 1988 году. Потом POSIX.2, но этого было все равно мало.
	
	В Европе был разработан стандарт X/Open.
	
	Это важно для нас, так как рынок ПО получил определенную свободу. Все это позволяет более эффективно разрабатывать ПО, привелкать больше людей к разработке.
	
	Inter Process Communication System5 включает в себя: сигналы, семафоры, программные каналы (им. и неим.), очереди сообщений, сегменты разделяемой памяти. Есть еще RPC [...].
	
	\section*{Сигналы}
	
	Сигналы информируют процессы и группы процессов.
	
	Сигнал - это программное средство информированния процессов о событиях, которые могут происходить внутри процессов (\textbf{синхронные} события) и  вне процессов (\textbf{ассинхронные} события).
	
	Все процессы, которые выполняются в системе, являются ассинхронными (выполняются с собственной скоростью независимо от того, как выполняются другие процессы).
	
	Как правило, получение некоторым процессом сигнала указывает ему завершить свое выполнение. Реакция процесса на получаемый сигнал зависит от того, как процесс определяет свою реакцию на получаемый сигнал: проигнорировать, реагировать на получаемый сигнал по умолчанию (так, как определено в системе) или  определить собственную реакцию на получаемый сигнал (для этого в коде процесса должен быть написан собственный обработчик сигнала).
	
	В классическом UNIX не могло быть больше 20 сигналов.
	
	\begin{lstlisting}
#define NSIG 20
#define SIGHUP 1
#define SIGINT 2
...
#define SIGKILL 9
...
#define SIGSEGV 11
#define SIGSYS 12
#define SIGPIPE 13
#define SIGALARM 14
#define SIGTERM 15
#define SIGUSR1 16
#define SIGUSR2 17
#define SIGCLD 18
#define SIGPWR 19
	\end{lstlisting}
	
	\begin{itemize}
		\item SIGHUP - сигнал разрыва с терминалом
		\item SIGINT - CTRL+C (завершение процесса)
		\item SIGKILL - уничтожение процесса системным вызовом kill
		\item SIGSEGV - нарушение сегментации (выход за пределы сегмента)
		\item SIGSYS - ошибка выполнения системного вызова
		\item SIGPIPE - запись в неименованный программный канал есть, чтения - нет
		\item SIGALARM - сигнал побудки
		\item SIGTERM
		\item SIGUSR1, SIGUSR2
		\item SIGCLD (в Linux - SIGCHLD) - сопровождает завершение процесса потомка
		\item SIGPWR - отключение питания (Для сохранения своей работоспособности ОС выполняет ряд действий: сохраняет важнейшие данные. При этом сигнал может перехватываться приложением)
	\end{itemize}

	О нарушениии сегментации: [картинка из методички про fork]. А в классическом UNIX было сегментированное управление памятью (???).
	
	Средством посылки сигналов в ОС UNIX являются два системных вызова: kill и signal.
	
	\subsection*{Системный вызов kill}

	\begin{lstlisting}
int kill(int pid, int sig);
	\end{lstlisting}
	\begin{lstlisting}
kill(getpid(), SIGALARM);
	\end{lstlisting}
	
	Сигнал побудки будет послан процессу, вызвавшему kill.
	
	!!! Для винды основным документом является MSDN. Для Linux - мануал.
	
	Если параметр $pid \le 1$, то сигнал будет послан группе процессов: если 0, то всем процессам с идентификатором группы, совпадающим с идентификатором группы процесса, который вызвал kill; если -1, то sig будет послан процессам, которые имеют тот же uid, что и процесс, вызвавший kill и т.д.
	
	\subsection*{Системный вызов signal}
	
	\begin{lstlisting}
void (*signal(int sig, void (*handler)(int)))(int);
	\end{lstlisting}

	signal не входит в POSIX, но входит в ANSI C. Поэтому signal не рекомендуется использовать в переносимых системах (аналог - sigaction).
	
	\begin{lstlisting}
#include <signal>

int main()
{
	void (*old_handler)(int) = signal(SIGINT, SIG_IGN);
	...
	signal(SIGINT, old_handler);
	...
}
	\end{lstlisting}

	\subsection*{sigaction}
	
	\begin{lstlisting}
int sigaction(int sig_numb, struct sigaction *action, struct sigaction *old_action);
	\end{lstlisting}

	В POSIX имеется два системных вызова: sigsetjump и siglongjmp [что делают].
	
	\section*{Программные каналы}
	
	mknod создает именованный программный канал. pipe - неименованный (не имеют имени, но имеют дескриптор struct inode).
	
	Именованным программным каналом может пользоваться программа, которая знает его имя [кажется, я написал не очень грамотно].
	
	[про буферизацию <<труб>>]
	
	Это связано с тем, что премещение страниц всегда оптимизировано.
	
	После создания трубы, если канал заполнен информацией, процесс записывающий будет приостановлен до тех пор, пока вся информация не будет прочитана и наоборот.
	
	На трубе реализовано взаимоисключение: нельзя читать, если в трубу пишут; нельзя писать, если из трубы читают.
\end{document}
