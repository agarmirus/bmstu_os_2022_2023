\documentclass[a4paper, 12pt]{report}

\usepackage[T2A]{fontenc}
\usepackage[utf8x]{inputenc}
\usepackage[english,russian]{babel}
\usepackage{amssymb,amsfonts,amsmath,mathtext,cite,enumerate,float}
\usepackage{pgfplots}
\usepackage{graphicx}
\usepackage{tocloft}
\usepackage{listings}
\usepackage[labelsep=period]{caption}

\makeatletter
\renewcommand{\@biblabel}[1]{#1.}
\makeatother

\usepackage{titlesec}
\titlespacing{\chapter}{0pt}{-\baselineskip}{\baselineskip}
\titleformat{\section}{\bfseries}{\thesection}{0.5 em}{}
\titlespacing*{\section}{0 pt}{\baselineskip}{\baselineskip}
\titlespacing*{\subsection}{0 pt}{\baselineskip}{\baselineskip}

\usepackage{geometry}
\geometry{left=2cm}
\geometry{right=1cm}
\geometry{top=2cm}
\geometry{bottom=2cm}

\usepackage{setspace}
\onehalfspacing

\renewcommand{\theenumi}{\arabic{enumi}}
\renewcommand{\labelenumi}{\arabic{enumi}}
\renewcommand{\theenumii}{.\arabic{enumii}}
\renewcommand{\labelenumii}{\arabic{enumi}.\arabic{enumii}.}
\renewcommand{\theenumiii}{.\arabic{enumiii}}
\renewcommand{\labelenumiii}{\arabic{enumi}.\arabic{enumii}.\arabic{enumiii}.}

\renewcommand{\cftchapleader}{\cftdotfill{\cftdotsep}} % for chapters

\renewcommand\bibname{Список использованных источников}

\captionsetup[table]{justification=raggedright} \captionsetup[figure]{justification=centering,labelsep=endash}
\DeclareCaptionLabelSeparator{dash}{~---~}
\captionsetup{labelsep=dash}

\usepackage{indentfirst}

\titleformat{\chapter}[hang]{\Huge\bfseries}{\thechapter}{1ex}{\Huge\bfseries}

\graphicspath{{images/}}%путь к рисункам

\newcommand{\floor}[1]{\lfloor #1 \rfloor}

\lstset{ %
	language=caml,                 % выбор языка для подсветки (здесь это С)
	basicstyle=\small\sffamily, % размер и начертание шрифта для подсветки кода
	numbers=left,               % где поставить нумерацию строк (слева\справа)
	numberstyle=\tiny,           % размер шрифта для номеров строк
	stepnumber=1,                   % размер шага между двумя номерами строк
	numbersep=5pt,                % как далеко отстоят номера строк от подсвечиваемого кода
	showspaces=false,            % показывать или нет пробелы специальными отступами
	showstringspaces=false,      % показывать или нет пробелы в строках
	showtabs=false,             % показывать или нет табуляцию в строках
	frame=single,              % рисовать рамку вокруг кода
	tabsize=2,                 % размер табуляции по умолчанию равен 2 пробелам
	captionpos=t,              % позиция заголовка вверху [t] или внизу [b] 
	breaklines=true,           % автоматически переносить строки (да\нет)
	breakatwhitespace=false, % переносить строки только если есть пробел
	escapeinside={\#*}{*)}   % если нужно добавить комментарии в коде
}

\pgfplotsset{width=0.85\linewidth, height=0.5\columnwidth}

\frenchspacing

\begin{document}
	\section*{Средства взаимодействия параллельных процессов}
	
	На лекции рассматривались семафоры Дейкстры. Семафор Дейкстры был определен как защещенная положительная переменная, на которой определены две операции, которые он определил как P(S) и V(S).
	
	P - захват семафора (декремент). Возможна, если значение семафора больше 0. Если опреацию выполнит ь нельзя, то процесс блокируется на семафоре.
	
	V - освобождение семафора (инкремент).
	
	P и V не являются эквивалентными понятиями к lock и unlock.
	
	Сама идея семафоров родилась на основе суммирования проблем взаимодействия процессов. Особенно - проблема активного ожидания на процессоре. Плата за это - переход в режим ядра (переключение аппаратного контекста минимум два раза) (только система может заблокировать и разблокировать процесс).
	
	Для структурирований действия с семафорами, системы поддерживают наборы считающих семафоров. Примеры - на основе задачи об обедающих философов.
	
	У семафора нет хозяина: семафор может освободить любой процесс.
	
	В ядре UNIX/Linux имеется таблица семафоров.
	
	[картинка]
	
	В этой таблице содержатся дескрипторы наборов семафоров.
	
	Библиотека - <sys/sem.h>.
	
	sem\_base - указатель на набор.
	
	Информация:
	
	\begin{itemize}
		\item Идентификатор - целое число. Присваивается процессом, который создал набор. Другие процессы по этому идентификатору могут получить дексриптор набора и оперировать этими наборами (!!!).
		
		\item UID создателя набора семафоров. Процесс, эффективный UID которого совпадает с UID создателя, может удалять и менять(???) набор.
		
		\item Права доступа (user, group, others)
		
		\item Количество семафоров в наборе
		
		\item Время изменения одного или нескольких семафоров
		
		\item Время изменения параметров набора(???)
		
		\item Указатель на массив семафоров
	\end{itemize}

	О каждом семафоре известно:
	
	%\begin{enumerate}
		[...]
	%\end{enumerate}

	На семафоре выделены следующие системные вызовы:
	
	semget() - создание набора семафоров
	
	semctl() - изменение параметров семафоров
	
	semop() - операция на семафоре
	
	[картинка с комментариями]
	
	В System V определено три операции.
	
	Операции sem\_op = 0 нет у Дейкстры (если семафор захвачен - блокируется процесс; без захвата).
	
	sem\_op > 0 - инкремент.
	
	На семафорах определены специальные флаги:
	
	\begin{itemize}
		\item IPC\_NOWAIT - [...]. Сделано для того, чтобы избежать блокировки всех процессов, находящихся в очереди для доступа к ресурсам, если захвативший ресурс процесс завершился аварийно или получил сигнал kill. Поскольку kill нельзя перехватить, то процесс не может освободить семафор, и все процессы в очереди буду заблоикрованы.
		
		\item SEM\_UNDO - указывает ядру на необходимость отслеживать [...]. При завершении процесса ядро ликвидирует сделанные процессом изменения. Добавлен по той же причине, что и IPC\_NOWAIT.
	\end{itemize}

	[пример]
	
	В данном примере создается набор с идентификатором 100 и количеством семафоров 2. Если удалось создать набор, вызывается sem\_op, в который передается массив структур.
	
	Данный пример также демонстрирует, что процесс не обязан освобождать уже захваченный семафор.
	
	\section*{Сегменты разделяемой памяти}
	
	Это средство передачи информации между процессами.
	
	[сопоставление программных каналов и сегментов разделяемой памяти]
	
	[таблица сегментов разделяемой памяти]
	
	shmget()
	
	shmctl()
	
	shmat()
	
	shmdt()
	
	На сегментах не определены методы взаимоисключения. Просто участок памяти. Просто пишем и читаем. Но для того, чтобы обеспечивать монопольный доступ к разделяемой памяти, используются семафор.
	
	Пример:
	
	\begin{lstlisting}
#include <string.h>
#include <sys/shm.h>

int main()
{
	int perms = S_IRWXU | S_IRWXG | S_IRWXO;
	int fd = shmget(100, 1024, IPC_CREAT | perms);
	if (fd == -1) {perror("shmget"); exit(1);}
	char *addr = (char *)shmat(fd, 0, 0);
	if (addr == (char *)-1) {perror("shmat"); exit(1);}
	strcpy(addr, "aaa");
	if (shmdt(addr) == -1) perror("shmdt");
	return 0;
}
	\end{lstlisting}

	[прочитать мануал к shmat и объяснить, почему второй аргумент - 0]
	
	ЛР:
	
	Создать не менее 3 потребителей и 3 производителей (через fork).
	
	Не меньше 3 писателей, не меньше 4 читателей.
	
	Производители пишут в буфер буквы английского алфавита по порядку. Потребители читают. Пишем с использованием указателей (не через индексы).
	
	Читатели и писатели азделяют одну единственную переменную. Писатели инкрементируют ее.
	
	Главное показать, что все работает правильно.
\end{document}
