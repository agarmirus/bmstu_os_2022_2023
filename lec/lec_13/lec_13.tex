\documentclass[a4paper, 14pt]{extreport}

\usepackage[T2A]{fontenc}
\usepackage[utf8]{inputenc}
\usepackage[english,russian]{babel}
\usepackage{amssymb,amsfonts,amsmath,mathtext,cite,enumerate,float}
\usepackage{pgfplots}
\usepackage{graphicx}
\usepackage{tocloft}
\usepackage{listings}
\usepackage{caption}
\usepackage{tempora}
\usepackage{titlesec}
\usepackage{setspace}
\usepackage{geometry}
\usepackage{indentfirst}
\usepackage{pdfpages}

\def\labelitemi{-}

\newcommand{\ssr}[1]{\begin{center}
		\LARGE\bfseries{#1}
	\end{center} \addcontentsline{toc}{chapter}{#1}  }

\makeatletter
\renewcommand{\@biblabel}[1]{#1.}
\makeatother

\titleformat{\chapter}[hang]{\LARGE\bfseries}{\hspace{1.25cm}\thechapter}{1ex}{\LARGE\bfseries}
\titleformat{\section}[hang]{\Large\bfseries}{\hspace{1.25cm}\thesection}{1ex}{\Large\bfseries}
\titleformat{name=\section,numberless}[hang]{\Large\bfseries}{\hspace{1.25cm}}{0pt}{\Large\bfseries}
\titleformat{\subsection}[hang]{\large\bfseries}{\hspace{1.25cm}\thesubsection}{1ex}{\large\bfseries}
\titlespacing{\chapter}{0pt}{-\baselineskip}{\baselineskip}
\titlespacing*{\section}{0pt}{\baselineskip}{\baselineskip}
\titlespacing*{\subsection}{0pt}{\baselineskip}{\baselineskip}

\geometry{left=3cm}
\geometry{right=15mm}
\geometry{top=2cm}
\geometry{bottom=2cm}

\onehalfspacing

\renewcommand{\theenumi}{\arabic{enumi}}
\renewcommand{\labelenumi}{\arabic{enumi}\text{)}}
\renewcommand{\theenumii}{.\arabic{enumii}}
\renewcommand{\labelenumii}{\asbuk{enumii}\text{)}}
\renewcommand{\theenumiii}{.\arabic{enumiii}}
\renewcommand{\labelenumiii}{\arabic{enumi}.\arabic{enumii}.\arabic{enumiii}.}

\renewcommand{\cftchapleader}{\cftdotfill{\cftdotsep}}

\captionsetup[figure]{justification=centering,labelsep=endash}
\captionsetup[table]{labelsep=endash,justification=raggedright,singlelinecheck=off}

\DeclareCaptionLabelSeparator{dash}{~---~}
\captionsetup{labelsep=dash}

\graphicspath{{images/}}%путь к рисункам

\newcommand{\floor}[1]{\lfloor #1 \rfloor}

\lstset{ %
	language=caml,                 % выбор языка для подсветки (здесь это С)
	basicstyle=\small\sffamily, % размер и начертание шрифта для подсветки кода
	numbers=left,               % где поставить нумерацию строк (слева\справа)
	numberstyle=\tiny,           % размер шрифта для номеров строк
	stepnumber=1,                   % размер шага между двумя номерами строк
	numbersep=5pt,                % как далеко отстоят номера строк от подсвечиваемого кода
	showspaces=false,            % показывать или нет пробелы специальными отступами
	showstringspaces=false,      % показывать или нет пробелы в строках
	showtabs=false,             % показывать или нет табуляцию в строках
	frame=single,              % рисовать рамку вокруг кода
	tabsize=2,                 % размер табуляции по умолчанию равен 2 пробелам
	captionpos=t,              % позиция заголовка вверху [t] или внизу [b] 
	breaklines=true,           % автоматически переносить строки (да\нет)
	breakatwhitespace=false, % переносить строки только если есть пробел
	escapeinside={\#*}{*)}   % если нужно добавить комментарии в коде
}

\pgfplotsset{width=0.85\linewidth, height=0.5\columnwidth}

\linespread{1.3}

\parindent=1.25cm

\frenchspacing

\begin{document}
	\chapter{Тупики в распределенных системах}
	
	[...]
	
	Транзакции в системе распределенной БД выполняются на нескольких сайтах и используют данные на этих сайтах. При этом объем данных распределяется между сайтами неравномерно. Кроме того, варьируется время выполнения на каждом сайте. В результате одна и та же транзакция может быть активной на одних сайтах и неактивной на других.
	
	Если на сайте находятся две конфликтующие транзакции, возможна ситуация, при которой одна из транзакций находится не в активном состоянии. То есть важно местоположение транзакций (\textbf{проблема местоположения транзакций, transaction location}).
	
	Для решения данной проблемы используется модель Daisy Chain. Она предполагает хранение доп информации о транзакции. При перемещении транзакции с одного сайта на другой предлагается хранить список необходимых таблиц и список сайтов [там еще что-то], а также список блокировок с типами.
	
	Вся дополнительная информация о транзакциях должна быть отправлена на все заинтересованные сайты для анализа.
	
	\textbf{Пример: } Chundy-Misza-Haas algorithm
	
	Процессы запрашивают сразу несколько необходимых им ресурсов, что уменьшает количество транзакций. Если при очередном запросе ресурс занят, то процесс генерирует специальное сообщение и посылает его другим взаимодействующим с ним процессам. В сообщении указывается три поля: номер процесса, отправляющего сообщение; номер процесса, отправляющего сообщение (да-да, то же самое); номер процесса, которому посылается сообщение.
	
	Процесс, получивший сообщение, проверяет, ждет ли он сам ресурс, запрашиваемый другим процессом. Если ждет, то во второе поле он записывает свой номер, а в третье поле записывает номер процесса, от которого ждет освобождение ресурса. Затем посылает сообщение дальше.
	
	В результате, если процесс получит сообщение обнаружит свой номер в первом и третьем поле, то тем самым он обнаружит, что система находится в тупике. Такое сообщение называется зондом.
	
	[картинка]
	
	А как блокированный процесс может что-то делать (анализировать, принимать, посылать)? Дополнительный поток, который может каким-то чудом выполнять работу.
	
	\chapter{Управление памятью}
	
	Речь идет об оперативной памяти.
	
	Память как ресурс неоднородна. В системе может рассматриваться иерархия памяти, в зависимости от близости к процессору.
	
	[что-то про программируемые процессоры и микрокоманды]
	
	!!! Процессор собственной памяти не имеет и постоянно обменивается информацией с оперативной памяти (считывает команды и данные из ОП и записывает данные в ОП).
	
	По архитектуре фон Неймана, процессор может выполнять только те программы, что находятся в ОП. [про работу программ].
	
	Вторичная память (флеш-память, SSD и тд). Несмотря на то, что ее физические принципы изменились, вторичная память остается блочной. Вторичная память является энергонезависимой и предназначена для долговременного хранения данных.
	
	У вторичной памяти есть еще одна задача: поддержка в современных системах пэйджинга. То есть компьютер имеет память, аналогичную дисковой, и дисковое адресное пространство делится на две неравные части. Большая часть используется для хранения данных, при этом имеются в виду т.н. обычные файлы (regular).
	
	Файл - поименованная совокупность данных (имеет имя). Эта совокупность может быть бессмысленной. Но чтобы можно было обратиться к этим данным, то эти данные должны быть идентифицироваться. В UNIX идентификатором файла является inode, но в системе файлы идентифицируются именем.
	
	Очевидно, что эта иерархия памяти, строящаяся по принципу близости к процессору, имеет определенные характеристики. В частности, память должна иметь то же быстродействие, что и процессор. Если это требование не будет выполняться, то отпадает смысл повышать производительность процессора.
	
	В современных системах имеется такая ситуация, при которой быстродействие ОП отстает на порядок от производительности процессора ввиду разных техник производства. Поэтому в процессорах есть кеши трех уровней. Кеш дает возможность решить проблему с отставанием.
	
	В современных системах есть одноуровневая память (файлы, отображаемые в память) --- для пэйджинга используется память файла.
	
	\section{Управление память в старых системах}
	
	Принято при управлении памятью рассматривать связанное (программа в памяти занимает непрерывное адресное пространство) и несвязанное (программа может занимать в память непрерывные участки) распределение памяти.
	
	Классификация:
	
	\begin{enumerate}
		\item Одиночное непрерывное распределение
		
		Речь идет об однопрограммных системах, в которых в каждый момент времени в памяти находится только одно приложение. При таком распределении находятся две программы: ОС и приложение. Оставшаяся часть --- свободная неиспользуемая память.
		
		[картинки]
		
		[что-то про DOS]
		
		\item Распределение памяти разделами
		
		Размеры разделов выбирались из наиболее часто встречающихся программ.
		
		[картинка]
		
		Данный метод крайне не эффективен.
		
		[про одну очередь к разделам]
		
		\item Динамическое определение размеров разделов
		
		[...]
		
		
	\end{enumerate}
\end{document}
