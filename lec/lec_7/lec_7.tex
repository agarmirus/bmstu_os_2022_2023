\documentclass[a4paper, 14pt]{report}

\usepackage{extsizes}
\usepackage[T2A]{fontenc}
\usepackage{listings}
\usepackage[utf8x]{inputenc}%включаем свою кодировку: koi8-r или utf8 в UNIX, cp1251 в Windows
\usepackage[english,russian]{babel}%используем русский и английский языки с переносами
\usepackage{amssymb,amsfonts,amsmath,mathtext,cite,enumerate,float} %подключаем нужные пакеты расширений
\usepackage{graphicx} %хотим вставлять в диплом рисунки?

\makeatletter
\renewcommand{\@biblabel}[1]{#1.} % Заменяем библиографию с квадратных скобок на точку:
\makeatother

\usepackage{titlesec}

\usepackage{geometry} % Меняем поля страницы
\geometry{left=3cm}% левое поле
\geometry{right=1cm}% правое поле
\geometry{top=2cm}% верхнее поле
\geometry{bottom=2cm}% нижнее поле

\renewcommand{\theenumi}{\arabic{enumi}}% Меняем везде перечисления на цифра.цифра
\renewcommand{\labelenumi}{\arabic{enumi}}% Меняем везде перечисления на цифра.цифра
\renewcommand{\theenumii}{.\arabic{enumii}}% Меняем везде перечисления на цифра.цифра
\renewcommand{\labelenumii}{\arabic{enumi}.\arabic{enumii}.}% Меняем везде перечисления на цифра.цифра
\renewcommand{\theenumiii}{.\arabic{enumiii}}% Меняем везде перечисления на цифра.цифра
\renewcommand{\labelenumiii}{\arabic{enumi}.\arabic{enumii}.\arabic{enumiii}.}% Меняем везде перечисления на цифра.цифра

\usepackage{indentfirst}

\titleformat{\chapter}[hang]{\Huge\bfseries}{\thechapter.}{1ex}{\Huge\bfseries}

\graphicspath{{images/}}%путь к рисункам

\begin{document}
	\section*{Взаимодействие параллельных процессов}
	
	Достоинство семафоров: устранение активного ожидания на процессоре (когда процессор тратит квант процессорного времени на проверку флага или переменной).
	
	Недостаток: переход в режим ядра (только ядро может заблокировать или разблокировать процесс).
	
	!!! В ЛР необходимо при реализации создавать набор из трех семафоров.
	
	Для решения проблемы структурирования была придумана концепция {\bf мониторов}.
	
	\subsection*{Мониторы}
	
	Мониторы были разработаны как средства более высокого уровня, чем т.н. примитивы ядра (примитив - не значит глупый, а значит низкоуровневое средство в распоряжении процессов).
	
	Расмотренные выше команды test-and-set для разных систем можно объединить под общими названиями \textbf{lock} и \textbf{unlock}:
	
	\begin{lstlisting}
lock() - unlock()
-----------------
wait() - post()
wait() - signal
	\end{lstlisting}

	\textbf{Монитор} - механизм, унифицирующий управление взаимоисключением (???)
	
	Монитор обозначается ключевым словом monitor, при этом монитор может предоставляться операционной системой или ЯП.
	
	Монитор - это набор процедур и данных, причем обращаться к данным монитора можно обращаться только с помощью процедур. Процесс, вызвавший процедуру монитора называется \textbf{процессом, находящимся в мониторе}, при этом монитор гарантирует, что в каждый момент времени процедура монитора может использовать только один процесс. Остальные процессы, заинтересованные в вызове процедуры монитора, ставятся в очередь к монитору.
	
	Как правило, монитор оперирует переменными типа conditional (условие) с помощью двух функций wait и signal (это системные вызовы). wait блокирует процесс, signal - разблокировывает его.
	
	Расмотрим три классических монитора: простой, кольцевой буфер, читатели-писатели.
	
	\textbf{Простой монитор} обеспечивает выделение определенного ресурса произвольному числу процессов.
	
	Код монитора (!!!):
	
	\begin{lstlisting}
resource: monitor;

var
	busy: logical;
	x: conditional;

procedure acquire;
	begin
		if busy then wait(x);
		busy := true;
	end;

procedure release;
	begin
		busy := false;
		signal(x);
	end;

begin
	busy = false;
end;
	\end{lstlisting}

	Когда к монитору обращается процесс для захвата ресурса, они вызывает функцию acquire. Если значение логической переменной busy - истина, то по переменной x выполняется системный вызов wait. В результате значение логической переменной не меняется.
	
	Если busy - ложь, то процесс, обратившийся к монитору с помощью acquire, получает доступ к ресурсу и продолжается без задержки. В результате, busy устанавливается значением истины. 
	
	Если процесс, который занимает ресурс, желает его освободить, то он вызывает процедуру монитора release и меняет значение busy на ложь, после этого вызывается функция signal, которая разблокирует другой процесс, который назодится в очереди к переменной типа собитие.
	
	Для каждой причины, по которой монитор переводится в режим ожидания (???), назначается своя переменная типа условие. Какждая переменная типа условие - это способ обозначения соответствующей очереди.
	
	Действия wait и signal подобны действиям P и V.
	
	!!! Семафоры всегда существуют в любой системе.
	
	\textbf{Доп.:} В винде и юникс есть семафоры и мьютексы (mutex).
	
	\subsection*{Кольцевой буфер}
	
	Кольцевой буфер решает задачу производства-потребителя.
	
	Для этой задачи характерно наличие \textbf{\textit{двух типов}} процессов: процессы-производители (могут только производить единицы данных и помещать их в буфер) и процессы-потребители (могут только выбирать данные из буфера).
	
	[код]
	
	Потребитель может взять только данные, которые есть в буфере. Если все ячейки свободные, то потребитель будет ждать производителя (на самом деле, это блокировка).
	
	\subsection*{Читатели-писатели}
\end{document}
