\documentclass[a4paper, 14pt]{extreport}

\usepackage[T2A]{fontenc}
\usepackage[utf8]{inputenc}
\usepackage[english,russian]{babel}
\usepackage{amssymb,amsfonts,amsmath,mathtext,cite,enumerate,float}
\usepackage{pgfplots}
\usepackage{graphicx}
\usepackage{tocloft}
\usepackage{listings}
\usepackage{caption}
\usepackage{tempora}
\usepackage{titlesec}
\usepackage{setspace}
\usepackage{geometry}
\usepackage{indentfirst}
\usepackage{pdfpages}

\def\labelitemi{-}

\newcommand{\ssr}[1]{\begin{center}
		\LARGE\bfseries{#1}
	\end{center} \addcontentsline{toc}{chapter}{#1}  }

\makeatletter
\renewcommand{\@biblabel}[1]{#1.}
\makeatother

\titleformat{\chapter}[hang]{\LARGE\bfseries}{\hspace{1.25cm}\thechapter}{1ex}{\LARGE\bfseries}
\titleformat{\section}[hang]{\Large\bfseries}{\hspace{1.25cm}\thesection}{1ex}{\Large\bfseries}
\titleformat{name=\section,numberless}[hang]{\Large\bfseries}{\hspace{1.25cm}}{0pt}{\Large\bfseries}
\titleformat{\subsection}[hang]{\large\bfseries}{\hspace{1.25cm}\thesubsection}{1ex}{\large\bfseries}
\titlespacing{\chapter}{0pt}{-\baselineskip}{\baselineskip}
\titlespacing*{\section}{0pt}{\baselineskip}{\baselineskip}
\titlespacing*{\subsection}{0pt}{\baselineskip}{\baselineskip}

\geometry{left=3cm}
\geometry{right=15mm}
\geometry{top=2cm}
\geometry{bottom=2cm}

\onehalfspacing

\renewcommand{\theenumi}{\arabic{enumi}}
\renewcommand{\labelenumi}{\arabic{enumi}\text{)}}
\renewcommand{\theenumii}{.\arabic{enumii}}
\renewcommand{\labelenumii}{\asbuk{enumii}\text{)}}
\renewcommand{\theenumiii}{.\arabic{enumiii}}
\renewcommand{\labelenumiii}{\arabic{enumi}.\arabic{enumii}.\arabic{enumiii}.}

\renewcommand{\cftchapleader}{\cftdotfill{\cftdotsep}}

\captionsetup[figure]{justification=centering,labelsep=endash}
\captionsetup[table]{labelsep=endash,justification=raggedright,singlelinecheck=off}

\DeclareCaptionLabelSeparator{dash}{~---~}
\captionsetup{labelsep=dash}

\graphicspath{{images/}}%путь к рисункам

\newcommand{\floor}[1]{\lfloor #1 \rfloor}

\lstset{ %
	language=caml,                 % выбор языка для подсветки (здесь это С)
	basicstyle=\small\sffamily, % размер и начертание шрифта для подсветки кода
	numbers=left,               % где поставить нумерацию строк (слева\справа)
	numberstyle=\tiny,           % размер шрифта для номеров строк
	stepnumber=1,                   % размер шага между двумя номерами строк
	numbersep=5pt,                % как далеко отстоят номера строк от подсвечиваемого кода
	showspaces=false,            % показывать или нет пробелы специальными отступами
	showstringspaces=false,      % показывать или нет пробелы в строках
	showtabs=false,             % показывать или нет табуляцию в строках
	frame=single,              % рисовать рамку вокруг кода
	tabsize=2,                 % размер табуляции по умолчанию равен 2 пробелам
	captionpos=t,              % позиция заголовка вверху [t] или внизу [b] 
	breaklines=true,           % автоматически переносить строки (да\нет)
	breakatwhitespace=false, % переносить строки только если есть пробел
	escapeinside={\#*}{*)}   % если нужно добавить комментарии в коде
}

\pgfplotsset{width=0.85\linewidth, height=0.5\columnwidth}

\linespread{1.3}

\parindent=1.25cm

\frenchspacing

\begin{document}
	\chapter{Продолжение предыдущей лекции}
	
	Очень важно понимать, что преобразование адресов выполняется на каждой команде, а то и несколько раз, особенно если в команде есть косвенная адресация.
	
	[картинка]
	
	Преобразование осталось, но изменило систему программирования. Появились адресно-независимые программы (позволило перемещать программы в памяти). Изменилось все.
	
	Когда перемещать программу? Два подхода: если в памяти для загрузки не удавалось найти раздела необходимого раздела, и [...].
	
	Увеличение размеров ОП привело к росту размеров программ. Таким образом, размещение программы единым сегментом стало затратным.
	
	Возникла идея разделения адресного пространства процесса на участки равного размера - страницы. Все страницы программы загружались в ОП. В результате, эта идея реализовывалась (первоначально) через создание таблицы страниц, в которой указывались смещения:
	
	[картинка]
	
	Сама таблица страниц находится в ОП.
	
	Появилось понятие фрейма - страница адресного пространства процесса вставляется в <<рамку>> сегмента.
	
	А может ли выполняться программа, которая в памяти находится не полностью? Данный вопрос привел к появлению виртуальной памяти.
	
	\chapter{Виртуальной памяти}
	
	Существует три схемы управления виртуальной памятью: выделение памяти страницами по запросу; выделение памяти сегментами по запросу; выделение памяти сегментами, поделенными на страницы, по запросу.
	
	Запрос возникает, когда процессор обращается к команде или данным, отсутствующим в физической памяти. Запросы реализованы в виде прерываний.
	
	\section{Страничная память}
	
	Когда процесс создается, первое, что делается, - это выделение страниц процессу. Дальше процесс начинает выполняться команда за командой. Но в какой-то момент процессор обращается к данным или команде, отсутствующим в памяти - возникает страничное прерывание.
	
	Чтобы процесс смог продолжить свое выполнение, страница должна быть загружена в память. При этом выполняется преобразование, при котором смещение берется из программы.
	
	Для каждого процесса системой выделяется виртуальное адресное пространство.
	
	Для того, чтобы загрузить виртуальную таблицу в физическую память, необходимо выполнять преобразования.
	
	Для выделения памяти страницами по запросу были предложены три вида: прямое отображение, ассоциативное и ассоциативно-прямое.
	
	\begin{enumerate}
		\item прямое
		
		[картинки]
		\item ассоциативное
		
		[картинки]
		
		Ассоциативная паять - память регистровая (потому и дорогая). Устроена таким образом, что за один такт происходит сравнение с всеми ключами и поиск нужной страницы. Ключом является смещение.
		
		[картинка со схемой совпадения]
		\item ассоциативно-прямое
		
		[картинки]
		
		Сначала страница ищется в кэше. Если не найдена, то происходит обращение к таблице страниц. При этом в кэше должны храниться адреса физические адреса, которым были недавние обращения. Должно выполняться замещение, если страница в кэше не найдено. В результате даже небольшой кэш обеспечивает до 98\% эффективных обращений.
	\end{enumerate}

	[...]
	
	\section{Алгоритмы вытеснения страниц}
	
	\begin{enumerate}
		\item Выталкивание случайной страницы
		
		Характеризуется малыми накладными расходами. Не является дискриминационным, так как любая страница может быть вытеснена, и вероятность вытеснения у каждой страницы одинаковой. Применяется редко.
		
		Может быть вытолкнута частная страница или только что загруженная.
		
		\item FIFO
		
		Для реализации каждой странице или приписывается временная метка, или организуется связный список страниц типа FIFO. Выталкивается страница, которая дольше всего находится в памяти. Может быть вытолкнута часто используемая страница.
	\end{enumerate}
\end{document}
