\documentclass[a4paper, 12pt]{report}

\usepackage[T2A]{fontenc}
\usepackage[utf8x]{inputenc}
\usepackage[english,russian]{babel}
\usepackage{amssymb,amsfonts,amsmath,mathtext,cite,enumerate,float}
\usepackage{pgfplots}
\usepackage{graphicx}
\usepackage{tocloft}
\usepackage{listings}
\usepackage[labelsep=period]{caption}

\makeatletter
\renewcommand{\@biblabel}[1]{#1.}
\makeatother

\usepackage{titlesec}

\usepackage{geometry}
\geometry{left=3cm}
\geometry{right=1cm}
\geometry{top=2cm}
\geometry{bottom=2cm}

\renewcommand{\theenumi}{\arabic{enumi}}
\renewcommand{\labelenumi}{\arabic{enumi}}
\renewcommand{\theenumii}{.\arabic{enumii}}
\renewcommand{\labelenumii}{\arabic{enumi}.\arabic{enumii}.}
\renewcommand{\theenumiii}{.\arabic{enumiii}}
\renewcommand{\labelenumiii}{\arabic{enumi}.\arabic{enumii}.\arabic{enumiii}.}

\renewcommand{\cftchapleader}{\cftdotfill{\cftdotsep}} % for chapters

\usepackage{indentfirst}

\titleformat{\chapter}[hang]{\Huge\bfseries}{\thechapter.}{1ex}{\Huge\bfseries}

\graphicspath{{images/}}%путь к рисункам

\newcommand{\floor}[1]{\lfloor #1 \rfloor}

\lstset{ %
	language=caml,                 % выбор языка для подсветки (здесь это С)
	basicstyle=\small\sffamily, % размер и начертание шрифта для подсветки кода
	numbers=left,               % где поставить нумерацию строк (слева\справа)
	numberstyle=\tiny,           % размер шрифта для номеров строк
	stepnumber=1,                   % размер шага между двумя номерами строк
	numbersep=5pt,                % как далеко отстоят номера строк от подсвечиваемого кода
	showspaces=false,            % показывать или нет пробелы специальными отступами
	showstringspaces=false,      % показывать или нет пробелы в строках
	showtabs=false,             % показывать или нет табуляцию в строках
	frame=single,              % рисовать рамку вокруг кода
	tabsize=2,                 % размер табуляции по умолчанию равен 2 пробелам
	captionpos=t,              % позиция заголовка вверху [t] или внизу [b] 
	breaklines=true,           % автоматически переносить строки (да\нет)
	breakatwhitespace=false, % переносить строки только если есть пробел
	escapeinside={\#*}{*)}   % если нужно добавить комментарии в коде
}

\pgfplotsset{width=0.85\linewidth, height=0.5\columnwidth}

\frenchspacing

\begin{document}
	[картинка]
	
	Традиционно, в UNIX ведется подсчет реального времени и времени с момента включения системы (по крайней мере, так было раньше).
	
	\section*{Что же определяется как время?}
	
	Любая развитая страна имеет службу точного времени. Этим занимаются специальные лаборатории. Момент подъема солнца на максимальную высоту - солнечный переход. Минимальный интервал между солнечными переходами - солнечный день. Была введена солнечная секунда (mean solar second).
	
	В 1948 году были изобретены атомные часы. На основе средней солнечной секунды было определено, что секунда - это время, за которое Цезий-133 совершает 9192631770 переходов.
	
	Самой большой объем программирования осуществляется для систем реального времени.
	
	Современные атомные часы работают на тулии. В России часы на тулии есть в Институте имени Лебедева.
	
	Значения точных времен усредняется и формируется International Atomic Time (TAI).
	
	Солнечный день в настоящее время изменяется, а солнечная секунда остается - возникают расхождения. Чтобы такого не возникало, было решено использовать т.н. потерянные секунды (если разница возрастает до 800 мс). Все компьютеры, в которых принимаются значения точного времени, должны подстраиваться.
	
	Universal Coordinated Time (UTC).
	
	[...]
	
	Алгоритмы:
	
	\begin{itemize}
		\item Кристиана;
		\item Беркли;
		\item Усредняющие;
	\end{itemize}

	При взаимодействии процессов важным моментом является обеспечение правильного порядка выполнения процессов - обеспечение прицинно-следственной связи (<<случилось до - случилось после>>). В соответствии с этим был предложен алгоритм Лампорта (алгоритм логических часов Лампорта).
	
	[картинка]
	
	Нарушается отношение, так как время отправки отличается от времени приема сообщения. Поэтому был предложен алгоритм Лампорта: [...]
	
	\section*{Алгоритмы взаимоисключения в распределенных системах}
	
	\begin{itemize}
		\item Централизованный алгоритм (алгоритм забияки) (самый надежный, если дополнен выбором нового координатора)
		
		Существует процесс-координатор, который координирует работой процессов. [...]
		
		\item Распределенный
		
		[...]
		
		\item Token Ring
	\end{itemize}

	\section*{Удаленный вызов процедур}
	
	[...]
	
	[картинка]
\end{document}
